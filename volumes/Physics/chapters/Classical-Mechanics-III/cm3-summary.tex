\chapter{Classical Mechanics III}

\section{Hamiltonian mechanics}
\textbf{Hamiltonian mechanics} is a reformulation of classical mechanics that provides deep insights into both classical and quantum systems. It is named after the Irish mathematician William Rowan Hamilton and is built upon the concept of the \textit{Hamiltonian function}, which describes the total energy of the system.

Hamiltonian mechanics provides a bridge between classical mechanics and quantum mechanics and forms the foundation of many modern physical theories.

\section{Key Concepts}
\begin{itemize}
    \item \textbf{Phase Space:} A 2$n$-dimensional space where each point represents the state of the system, with generalized coordinates $q_i$ and conjugate momenta $p_i$.
    \item \textbf{Hamiltonian Function:} Denoted by $H(q_i, p_i, t)$, the Hamiltonian typically represents the total energy of the system (kinetic + potential energy).
    \item \textbf{Canonical Variables:} The generalized coordinates $q_i$ and conjugate momenta $p_i$ form pairs of canonical variables.
\end{itemize}

\section{Hamilton's Equations of Motion}
Hamilton's equations are a set of first-order differential equations that describe the time evolution of a system:
\[
\dot{q}_i = \frac{\partial H}{\partial p_i}, \quad \dot{p}_i = -\frac{\partial H}{\partial q_i}
\]
where:
\begin{itemize}
    \item $\dot{q}_i = \frac{dq_i}{dt}$ is the time derivative of the generalized coordinate.
    \item $\dot{p}_i = \frac{dp_i}{dt}$ is the time derivative of the conjugate momentum.
\end{itemize}

\section{Derivation from the Principle of Least Action}
Hamiltonian mechanics can be derived from the \textit{principle of least action}, where the action $S$ is given by:
\[
S = \int_{t_1}^{t_2} \left( \sum_{i=1}^{n} p_i \dot{q}_i - H \right) dt
\]
The equations of motion are obtained by requiring that $S$ be stationary ($\delta S = 0$).

\section{Examples}
\subsection*{1. Simple Harmonic Oscillator}
For a simple harmonic oscillator with mass $m$ and spring constant $k$, the Hamiltonian is:
\[
H = \frac{p^2}{2m} + \frac{1}{2} k q^2
\]
Hamilton's equations are:
\[
\dot{q} = \frac{\partial H}{\partial p} = \frac{p}{m}, \quad \dot{p} = -\frac{\partial H}{\partial q} = -kq
\]

\subsection{2. Free Particle}
For a free particle of mass $m$, the Hamiltonian is purely kinetic:
\[
H = \frac{p^2}{2m}
\]
The equations of motion are:
\[
\dot{q} = \frac{p}{m}, \quad \dot{p} = 0
\]

\section{Canonical Transformations}
A \textit{canonical transformation} preserves the form of Hamilton's equations. If $(q_i, p_i) \to (Q_i, P_i)$ is a transformation such that the new variables also satisfy Hamilton's equations, then the transformation is canonical. 

Generating functions are often used to find such transformations, and they satisfy conditions like:
\[
p_i dq_i - H dt = P_i dQ_i - K dt
\]

\section{Poisson Brackets}
The Poisson bracket of two functions $f$ and $g$ in phase space is defined as:
\[
\{f, g\} = \sum_{i=1}^{n} \left( \frac{\partial f}{\partial q_i} \frac{\partial g}{\partial p_i} - \frac{\partial f}{\partial p_i} \frac{\partial g}{\partial q_i} \right)
\]
Hamilton's equations can be written in terms of Poisson brackets as:
\[
\dot{f} = \{f, H\} + \frac{\partial f}{\partial t}
\]

\section{Advantages of Hamiltonian Mechanics}
\begin{itemize}
    \item Provides a more symmetric formulation compared to Lagrangian mechanics.
    \item Offers a natural framework for the transition to quantum mechanics (via quantization of the Hamiltonian).
    \item Facilitates the study of conserved quantities and symmetries through Noether's theorem.
    \item Useful in systems with constraints and in statistical mechanics.
\end{itemize}

\section{Applications}
\begin{itemize}
    \item \textbf{Classical Mechanics:} Analyzing complex systems such as planetary motion and rigid body dynamics.
    \item \textbf{Quantum Mechanics:} The Hamiltonian operator plays a central role in the Schrödinger equation.
    \item \textbf{Statistical Mechanics:} The Hamiltonian defines the energy of microstates in a system.
    \item \textbf{Control Theory and Optics:} Hamiltonian systems are used in optimal control and ray optics.
\end{itemize}

\section{Conclusion}
Hamiltonian mechanics is a powerful and elegant reformulation of classical mechanics that emphasizes energy and phase space. Its importance extends beyond classical physics, providing the foundation for quantum mechanics and influencing many areas of theoretical physics.



\section{Port-Hamiltonian systems}
\textbf{Port-Hamiltonian systems} (PHS) extend classical Hamiltonian mechanics to encompass open systems with inputs and outputs. They are widely used in control theory and multi-physics modeling, as they naturally handle energy exchange with the environment.

A Port-Hamiltonian system describes the flow of energy in and out of a system, where the Hamiltonian function represents the total stored energy.

\section{Key Concepts}
\begin{itemize}
    \item \textbf{State Variables:} The system is characterized by state variables $x \in \mathbb{R}^n$, which typically include positions and momenta.
    \item \textbf{Hamiltonian Function:} The Hamiltonian $H(x)$ represents the total energy (sum of kinetic and potential energy).
    \item \textbf{Port Variables:} Inputs $u$ and outputs $y$ represent energy flow through the system's ports.
\end{itemize}

\section{Mathematical Formulation}
A Port-Hamiltonian system is described by:
\[
\dot{x} = \left( J(x) - R(x) \right) \frac{\partial H}{\partial x} + G(x) u
\]
\[
y = G^\top(x) \frac{\partial H}{\partial x}
\]
where:
\begin{itemize}
    \item $x \in \mathbb{R}^n$: State vector.
    \item $H(x)$: Hamiltonian (total energy).
    \item $J(x)$: Skew-symmetric interconnection matrix (describes energy-conserving dynamics).
    \item $R(x)$: Symmetric positive semi-definite matrix (describes energy dissipation).
    \item $G(x)$: Input matrix (describes how inputs affect the system).
    \item $u$: Input vector (external forces, voltages, etc.).
    \item $y$: Output vector (conjugate to $u$, representing power flow).
\end{itemize}

\section{Energy Balance}
The energy balance of a port-Hamiltonian system is given by:
\[
\frac{dH}{dt} = -\frac{\partial H}{\partial x}^\top R(x) \frac{\partial H}{\partial x} + y^\top u
\]
This shows that the change in stored energy is due to both dissipation (first term) and external power input (second term).

\section{Examples}
\subsection{1. Electrical Circuit}
Consider an $LC$ circuit with inductance $L$ and capacitance $C$. The state variables are the current $i$ through the inductor and the charge $q$ on the capacitor:
\[
H(q, i) = \frac{1}{2} \left( \frac{q^2}{C} + L i^2 \right)
\]
The Port-Hamiltonian representation is:
\[
\begin{pmatrix} \dot{q} \\ \dot{i} \end{pmatrix} = \begin{pmatrix} 0 & 1 \\ -1 & 0 \end{pmatrix} \begin{pmatrix} \frac{\partial H}{\partial q} \\ \frac{\partial H}{\partial i} \end{pmatrix} + \begin{pmatrix} 0 \\ 1 \end{pmatrix} u
\]
\[
y = \begin{pmatrix} 0 & 1 \end{pmatrix} \begin{pmatrix} \frac{\partial H}{\partial q} \\ \frac{\partial H}{\partial i} \end{pmatrix}
\]

\subsection{2. Mechanical System}
A mass-spring-damper system can be modeled using the state variables position $q$ and momentum $p$. The Hamiltonian is:
\[
H(q, p) = \frac{p^2}{2m} + \frac{1}{2} k q^2
\]
The dynamics can be written in Port-Hamiltonian form, considering external forces.

\section{Properties}
\begin{itemize}
    \item \textbf{Passivity:} Port-Hamiltonian systems are inherently passive, meaning they do not generate energy on their own.
    \item \textbf{Modularity:} Components can be connected easily, preserving energy balance.
    \item \textbf{Stability:} The Hamiltonian function often serves as a Lyapunov function, aiding in stability analysis.
\end{itemize}

\section{Canonical Transformations}
Port-Hamiltonian systems allow for canonical transformations, where the structure is preserved under changes of variables, ensuring that energy balance is maintained.

\section*{Control of Port-Hamiltonian Systems}
Control design for Port-Hamiltonian systems often focuses on shaping the Hamiltonian or modifying the interconnection and dissipation matrices. Common approaches include:
\begin{itemize}
    \item \textbf{Energy Shaping:} Redefining the Hamiltonian to achieve desired behavior.
    \item \textbf{Dissipation Injection:} Adding damping to improve stability.
\end{itemize}

\section{Applications}
\begin{itemize}
    \item \textbf{Electrical Systems:} Modeling power networks, circuits, and electromagnetic devices.
    \item \textbf{Mechanical Systems:} Analyzing robotic arms, vehicles, and mechanical linkages.
    \item \textbf{Multi-Physics Systems:} Coupled domains like electromechanical or thermo-mechanical systems.
    \item \textbf{Control Theory:} Design of controllers for energy-based systems.
\end{itemize}

\section{Conclusion}
Port-Hamiltonian systems provide a unified framework for modeling and controlling physical systems across multiple domains. By emphasizing energy flow and passivity, they ensure modularity, stability, and robustness in the analysis and design of complex systems.

