\documentclass{article}
\usepackage[a4paper,margin=1in]{geometry}
\usepackage{tikz}
\usepackage[T1]{fontenc}
\usepackage[utf8]{inputenc}
\usepackage{lmodern}
\pagestyle{empty}

\usepackage{xfp}
\usepackage{xparse}

\NewDocumentEnvironment{timeline}{O{14} O{24} O{250} O{-3000} O{2025}}{% width, height, year step, start year, end year
  % Begin group
  \begingroup

  % Local variables
  \def\tlwidth{#1}%
  \def\tlheight{#2}%
  \def\tlyearstep{#3}%
  \def\tlstart{#4}%
  \def\tlend{#5}%
  \def\verticallinepos{0.5}%
  \pgfmathtruncatemacro{\tlsecondstep}{\tlstart + \tlyearstep}
  \pgfmathtruncatemacro{\numticks}{(\tlend - \tlstart)/\tlyearstep}

  \begin{tikzpicture}

  % Vertical axis
  \draw[very thick] (\verticallinepos,0) -- (\verticallinepos,\tlheight);
  \node[rotate=90] at (0, 0.5*\tlheight) {Time};

  % Loop from i = 0 to numticks
  \foreach \i in {0,...,\numticks} {
  % Compute current year
  \pgfmathtruncatemacro{\year}{\tlstart + \i * \tlyearstep}

  % Calculate position (float)
  \pgfmathsetmacro{\pos}{(\tlend - \year)/(\tlend - \tlstart) * \tlheight}

  % Only draw if pos is inside the bounds
  \pgfmathparse{(\pos >= 0) && (\pos <= \tlheight)}
  \ifnum\pgfmathresult>0
    % Prepare label
    \pgfmathtruncatemacro{\absyear}{abs(\year)}
    \ifnum\year<0
      \def\labeltext{\absyear~BCE}
    \else
      \def\labeltext{\absyear~CE}
    \fi

    % Draw the tick
    \draw[gray] (\verticallinepos,\pos) -- (\tlwidth,\pos) node[right] {\small \labeltext};
  \fi
  }

  % A period the object with constant influence
  \def\timelineevent##1##2##3##4##5{% year start, year end, width, name, distance from left line
    \pgfmathsetmacro{\ystart}{(\tlend - ##1)/(\tlend - \tlstart) * \tlheight}
    \pgfmathsetmacro{\yend}{(\tlend - ##2)/(\tlend - \tlstart) * \tlheight}
    \pgfmathsetmacro{\ytext}{(\ystart + \yend)/2}
    \pgfmathsetmacro{\xtext}{##5 + (##3/2)}

    \draw[fill=yellow!50] (##5,\ystart) rectangle (##5+##3,\yend);
    \node[rotate=90] at (\xtext,\ytext) {##4};
  }%

  % A period of change of the object influence
  \def\timelineeventinfluencechange##1##2##3##4##5{% year start, year end, start width, end width, start distance from left line
    \pgfmathsetmacro{\ystart}{(\tlend - ##1)/(\tlend - \tlstart) * \tlheight}
    \pgfmathsetmacro{\yend}{(\tlend - ##2)/(\tlend - \tlstart) * \tlheight}
    \pgfmathsetmacro{\xbottomstart}{##5 + (##3-##4)/2}

    \filldraw[fill=blue!30]
      (##5,\ystart) -- (##5+##3,\ystart)
      -- (\xbottomstart + ##4,\yend) -- (\xbottomstart,\yend)
      -- cycle;
  }%

  \def\timelineeventsplit##1##2##3##4##5##6##7{% year start, year end, start width, left width, right width, start distance from left line, distance between left and right
    \pgfmathsetmacro{\ystart}{(\tlend - ##1)/(\tlend - \tlstart) * \tlheight}
    \pgfmathsetmacro{\yend}{(\tlend - ##2)/(\tlend - \tlstart) * \tlheight}
    \pgfmathsetmacro{\xmiddle}{##6+##3/2}
    \pgfmathsetmacro{\xleftbottomstart}{\xmiddle-##7/2-##4}
    \pgfmathsetmacro{\xrightbottomstart}{\xmiddle+##7/2}

    \filldraw[fill=blue!30] % Left part of split
      (##6,\ystart) -- (\xmiddle,\ystart)
      -- (\xleftbottomstart + ##4,\yend) -- (\xleftbottomstart,\yend)
      -- cycle;
    \filldraw[fill=blue!30] % Right part of split
      (\xmiddle,\ystart) -- (##6+##3,\ystart)
      -- (\xrightbottomstart + ##5,\yend) -- (\xrightbottomstart,\yend)
      -- cycle;
  }%
}
{%
  \end{tikzpicture}
  \endgroup % End local scope
}

\begin{document}

\iffalse

Time line and chapter for:
Europe and Middle East;
Asia excluding Middle East;
Oceania;
Sub-Saharan Africa;
Americas;

%https://usefulcharts.com/products/timeline-of-world-history?srsltid=AfmBOopS9m5ZgO2ZgwRIQmrdXolLk8MB2rmKRCYCXRPoW1JgQ21a_9G0

I also want to include:
major bettles and events;
wide adoption of technology, like paper, printing press, computer...;


There should a environmets or command that simplifies creating the time lines.
The user should be able to define a time line where the user can start, stop, increase 
or decrease the time line or split it.
The user can speciy the coller and the text it should display.

Environment:
- fixed 15 column/timelines.
- The width of a timeline is given in mm
- The location/which column and start end is given in the timeline.
- If the time line need to change the column then a new time line needs to defined.
- To connect time lines you have "connect" or "transition" which is a dashed line.
- You can define a split given the start and finish.


\fi

\section{Oveview of World History}

\begin{timeline}[14][24][250][-3300][2025]
  \timelineevent{-3000}{-1000}{1}{\tiny Ancient Egypt}{1}
  \timelineevent{1760}{1840}{1}{\tiny Industrial Rev.}{6}
\end{timeline}


\section{World Pre-Bronze Age}

\section{European and Middle Eastern Bronze Age}

\section{European and Middle Eastern Classic Antiquity}
\begin{timeline}[13][20][50][-600][500]

  % Roman empire
  \timelineevent{-600}{-250}{0.5}{\tiny Roman Kingdom}{3+1.5/2+1/2}
  \timelineeventinfluencechange{-270}{-250}{0.5}{1.5}{3+1.5/2+1/2}
  \timelineevent{-250}{-40}{1.5}{\small Roman Republic}{3+1.5/2}
  \timelineeventinfluencechange{-40}{-20}{1.5}{3}{3+1.5/2}
  \timelineevent{-20}{280}{3}{Roman Empire}{3}
  \timelineeventsplit{280}{300}{3}{1}{1}{3}{2}
  \timelineevent{300}{480}{1}{Western Empire}{2.5}
  \timelineevent{300}{500}{1}{Eastern Empire}{5.5}

\end{timeline}


\section{European Dark Ages}

\section{European Renesance}

\section{Americas Pre-Colonalisation}

\section{Sub-sahara Pre-Colonalisation}

\section{World Colonalisation and Pre-industrialistation}

\section{World Industrialization and End of Colonalism}

\section{World Technological Age}


\end{document}

