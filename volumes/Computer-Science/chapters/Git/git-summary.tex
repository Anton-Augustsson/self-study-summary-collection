\chapter{Git}

\section*{Abstract}
Git is a distributed version control system designed to handle everything from small to large projects efficiently. This document provides a detailed and fundamental explanation of how Git works, including its architecture, core concepts, and the mechanics behind its operations.

\section{Introduction}
Git is a powerful tool for managing source code and collaborating on software development projects. Unlike centralized version control systems, Git is distributed, meaning each developer has a complete copy of the repository, including its history. This section introduces the architecture of Git and explains its core concepts.

\section{Key Concepts in Git}

\subsection{Snapshots vs. Deltas}
Unlike many version control systems that track changes (deltas) between file versions, Git records the state of the repository at each commit. Each commit in Git is a snapshot of the project at a given point in time. If a file remains unchanged, Git stores a reference to the previous file rather than duplicating it.

\subsection{Three States in Git}
Git operates on three primary states:
\begin{itemize}
    \item \textbf{Working Directory:} The local directory where files are edited.
    \item \textbf{Staging Area:} An index where changes are staged before committing.
    \item \textbf{Repository:} The .git directory where committed snapshots and metadata are stored.
\end{itemize}

\subsection{Git's Data Model}
Git's data model is built on three core components:
\begin{itemize}
    \item \textbf{Blobs:} Store the content of files.
    \item \textbf{Trees:} Represent directory structures and reference blobs.
    \item \textbf{Commits:} Point to trees and contain metadata, such as author, timestamp, and parent commits.
\end{itemize}

\section{How Git Works}

\subsection{Commits and Hashes}
Each commit in Git is identified by a unique SHA-1 hash, which is generated based on the commit's content, including:
\begin{itemize}
    \item The tree object representing the project's directory structure.
    \item The parent commit(s).
    \item Metadata such as author, timestamp, and commit message.
\end{itemize}
This ensures that any change to a commit's content results in a completely new hash, maintaining integrity.

\subsection{Branches}
Branches are pointers to specific commits. The default branch in Git is \texttt{main} (formerly \texttt{master}).
\begin{itemize}
    \item Creating a branch creates a new pointer to a commit.
    \item Switching branches updates the working directory to match the snapshot of the commit the branch points to.
    \item Merging combines changes from one branch into another.
\end{itemize}

\subsection{Staging and Committing}
\begin{enumerate}
    \item \textbf{Staging:} Use \texttt{git add} to move changes to the staging area.
    \item \textbf{Committing:} Use \texttt{git commit} to save a snapshot of staged changes. Git creates a commit object that references the current tree and parent commit(s).
\end{enumerate}

\subsection{Distributed Architecture}
Each Git repository is complete and self-contained. This allows developers to work offline and perform all operations locally. Collaboration is achieved through pushing and pulling changes to/from remote repositories.

\section{Git Workflow}

\subsection{Cloning a Repository}
To start working on a project, clone the repository using:
\begin{verbatim}
    git clone <repository_url>
\end{verbatim}

\subsection{Typical Workflow}
A typical workflow in Git includes the following steps:
\begin{enumerate}
    \item \textbf{Modify Files:} Make changes in the working directory.
    \item \textbf{Stage Changes:} Use \texttt{git add} to stage the changes.
    \item \textbf{Commit Changes:} Create a snapshot using \texttt{git commit}.
    \item \textbf{Push Changes:} Share changes with a remote repository using \texttt{git push}.
    \item \textbf{Pull Updates:} Fetch and integrate changes from the remote repository using \texttt{git pull}.
\end{enumerate}

\section{Common Git Commands}
\begin{itemize}
    \item \texttt{git status} - Check the status of files in the working directory and staging area.
    \item \texttt{git diff} - View changes in the working directory or staging area.
    \item \texttt{git log} - View the history of commits.
    \item \texttt{git branch} - Manage branches.
    \item \texttt{git merge} - Combine branches.
    \item \texttt{git rebase} - Reapply commits on top of another base.
\end{itemize}

\section{Conclusion}
Git is a robust and versatile version control system that underpins modern software development workflows. Its distributed architecture, snapshot-based model, and powerful branching capabilities make it an essential tool for developers.

\section*{References}
\begin{itemize}
    \item \href{https://git-scm.com/}{Git Official Documentation}
    \item \href{https://www.atlassian.com/git/tutorials}{Atlassian Git Tutorials}
\end{itemize}