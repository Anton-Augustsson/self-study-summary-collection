\chapter{Digital Circuits}
\section{Combinatorial and Sequential Networks}
\begin{figure}[h]
    \vspace{10mm}
    \centering
    \includegraphics[width=10cm, height=6cm]{\pathICT/images/logic-gates.jpeg}
    \caption{Logic Gates. From \cite{}}
\end{figure}

\textbf{Bolean algebra}
(or: $+$), (and: $\cdot$)
\[
\begin{aligned}
  x + x       &= x \\
  x \cdot x   &= x \\
  &\quad   \\
  x + x'      &= 1 \\
  x \cdot x'  &= 0 \\
  &\quad   \\
  x + 1       &= 1 \\
  x \cdot 0   &= 0 \\
  &\quad   \\
  x + 0       &= x \\
  x \cdot 1   &= x \\
\end{aligned} \qquad
\begin{aligned}
  (x')'       &= x \\
  &\quad   \\
  x + (y + x) &= (x +y) + z \\
  x(yz)       &= (xy)z \\
  &\quad   \\
  x(y+z)      &= xy+xz \\
  x + yz      &= (x+y)(x+z)   \\
  x + xy      &= x  \\
  &\quad   \\
  xy+\overline{x}z &= xy+\overline{x}z +yz  \\
  (x+y)(\overline{x}+z) &= (x+y)(\overline{x}+z)(y+z)   \\
  \overline{(x+y)} &= \overline{x}\overline{y}  \\
  \overline{(xy)} &= \overline{x}+\overline{y}  \\
\end{aligned}
\] 

\newpage
\subsection{Combinational}
$f(x_6,x_5,x_4,x_3,x_2,x_1,x_0) = x_6x_4(x_5+x_3+x_0) + x_2x_1$
\begin{figure}[h]
    \centering
    \includegraphics[width=6cm]{\pathICT/images/bolean-factorization.png}
    \caption{Bolean factorization. From \cite{}}
\end{figure}

\begin{figure}[h]
    \centering
    \includegraphics[width=5cm]{\pathICT/images/kmap-patterns.png}
    \caption{KMap patters. From \cite{}}
\end{figure}
KMaps works from 4 variables.

\textit{Delays}, there is a small delay for each gate.

%implisent and explicet multiplicationo
%

\begin{figure}[h]
    \centering
    \includegraphics[width=12cm]{\pathICT/images/logic-families.png}
    \caption{Logic families. From \cite{}}
\end{figure}

\newpage
\textbf{Example: design circuit}
\begin{equation}
    f(x_3,x_2,x_1,x_0) = \sum(0,2,4,5,6,7) + d(8,10,12,14) \;\; \text{ - sum-of-products}
\end{equation}

\[
\begin{aligned}
     0 &= 0000 \\
     2 &= 0010 \\
     4 &= 0100 \\
     5 &= 0101 \\
     6 &= 0110 \\
     7 &= 0111 \\
     \\
     8 &= 1000 \\
    10 &= 1010 \\
    12 &= 1100 \\
    14 &= 1110 \\
\end{aligned} \qquad\qquad
\begin{aligned}
\begin{karnaugh-map}
   \manualterms{1,0,1,0,1,1,1,1,-,0,-,0,-,0,-,0}
   \implicant{4}{6}
   \implicantedge{0}{8}{2}{10}
\end{karnaugh-map}
\end{aligned}
\]

\begin{equation}
    f= \overline{x_0} + \overline{x_3}x_2    
\end{equation}

\begin{figure}[h]
    \centering
    \includegraphics[width=8cm]{\pathICT/images/example-cobinational-lodgic.pdf}
    \caption{lodgic Combinational lodgic.}
\end{figure}
 
%why is it slower to have a single gate with multiple input?

\newpage
\subsection{Sequential}
Remembers the state what happend before.

\begin{figure}[h]
    \centering
    \includegraphics[width=12cm, height=6cm]{\pathICT/images/sequential-circuits-moore-type.png}
    \caption{Sequential circuits moore type. From \cite{}}
\end{figure}

\begin{figure}[h]
    \vspace{10mm}
    \centering
    \includegraphics[width=10cm, height=6cm]{\pathICT/images/sequential-circuits-mealy-type.png}
    \caption{Sequential circuits mealy type. From \cite{}}
\end{figure}

\newpage
\subsubsection{Memory}
\begin{table}[h!]
    \centering
    \begin{tabular}{ | m{5cm} | m{5cm} | }
        \hline
        Description & Circuit \\
        \hline
        SR-latch (memory cell)    
        &
        \includegraphics[width=0.3\textwidth]{\pathICT/images/SR_latch.png} \\
        \hline
        Gated SR-latch
        &
        \includegraphics[width=0.3\textwidth]{\pathICT/images/Gated_SR_Latch.png} \\
        \hline
        Gated D-latch
        &
        \includegraphics[width=0.3\textwidth]{\pathICT/images/Gated_D_latch.png} \\
        \hline
        D Flip Flops
        &
        \includegraphics[width=0.3\textwidth]{\pathICT/images/D_Flip_Flop.png} \\
        \hline
    \end{tabular}
    \caption{Memory. From \cite{}}\label{tbl:memory}
\end{table}

\newpage
\subsubsection{Timind diagram}
\begin{figure}[h]
    \vspace{10mm}
    \centering
    \includegraphics[width=10cm]{\pathICT/images/timing-diagram.pdf}
    \caption{Timing diagram.}
\end{figure}


\subsubsection{State diagram}
\textbf{State code}
\begin{center}
\begin{tabular}{ | c | c c | }
    \hline
 State name & $q_1$ & $q_0$  \\ 
    \hline
 S0 & 0 & 0 \\
    \hline
 S1 & 0 & 1 \\
    \hline
 S2 & 1 & 0 \\
    \hline
 S3 & 1 & 1 \\
    \hline
\end{tabular}
\end{center}


\textbf{State Table}
In sequential circuits we use \textit{state table} instead of \textit{truth table}.
ex of state table format:
\begin{center}
\begin{tabular}{ | c | c | c | c | }
    \hline
 Previus state & Input & Next state & Output \\ 
    \hline
 0 0 & 0 & 0 0 & 0 \\ 
    \hline
 0 0 & 1 & 0 1 & 0 \\ 
    \hline
 0 1 & 0 & 0 1 & 1 \\ 
    \hline
 0 1 & 1 & 1 0 & 1 \\ 
    \hline
\end{tabular}
\end{center}

\newpage
\textbf{Moore-circuit}  
\begin{figure}[h]
    \centering
    \includegraphics[width=5cm]{\pathICT/images/state-diagram-moore.png}
    \caption{State diagram moore. From \cite{}}
\end{figure}
Note: in more machine we need a end state for output 1

\textbf{Mealy-circuit}  
\begin{figure}[h]
    \centering
    \includegraphics[width=5cm]{\pathICT/images/state-diagram-mealy.png}
    \caption{State diagram mealy. From \cite{}}
\end{figure}
Note: in mealy machine we dont need that.




%mux memory element?
%reliable latch, ltaches
%2gate stratage only allow one 
%finate state mashine (FSM)
%state transition diagram
%(1) one input one output
%(2) ?all input and output

\newpage
\subsection{Shift registers}
Each pit is prosest simultaniusly. Makes much simpler circuit.
\begin{figure}[h]
    \centering
    \includegraphics[width=10cm]{\pathICT/images/shift-register.png}
    \caption{Shift register. From \cite{}}
\end{figure}

\subsection{Multiplixer}
$f(A,B,C)=\sum(0,1,5,6)$
\begin{figure}[h]
    \centering
    \includegraphics[width=5cm]{\pathICT/images/mux.png}
    \caption{Mux. From \cite{}}
\end{figure}



