\chapter{Rhetoric}

\section{Classical Rhetoric}
Aristotle

Forensics
Epideictic
Symbouleutikon, deliberative rhetoric


\subsection{Ethos}
Convince your audience of your credibility.

\subsection{Pathos}
Emotion.

\subsection{Logos}
Logic and reason.


\section{Giving an Informative Speech}
%https://www.youtube.com/watch?v=Unzc731iCUY
\subsection{How To Start?}
\begin{itemize}
  \item[-] Don't start your presentations with a joke. People usually need some time to adjust to your speaking patterns and to you as a person.
  \item[+] Do start with an empowerment promise of what people will know once listening to your speech.  
\end{itemize}

\subsection{Samples}
The audience will not be focused of the speech entirety, that is why it's important to cycle threw the point you want to make about three times.
% TODO: how should this be done

It's also critical to build a fence around the idea presented. It should be clear how your idea differs from other similar ideas.
So not to be confused by something else.

Verbal punctuation provides the listener time to reflect and let the message sink in. A large room filled with people that is completely silent,
also make the audience more alert, as each individual feel an urge do something for this sometimes seemingly awkward silence. However, it's only last
about few seconds, not making the audience reacting negatively or awkwardly towards the verbal punctuation and the speaker. Additionally, the verbal 
punctuation provides a entry point for those in the audience who have drowsed off.

Another way to get people back onboard with the speech is to ask a question. You should only wait for about 7 seconds before moving on and never 
insult the audience if they can't provide an answer. The question is either too easy, making it embarking to answer or to difficult so no one can 
answer. By insulting the audience of an ill-posed question they will just get annoyed by you. They will be more active but at the cost of disliking
you.

\subsection{Time and Place}
Too early, people are sleepy, too late and people tiered, right after a meal and people want their siesta.
This usually results the optimal time is before lunch. The usual arises from understanding that different culture 
have different habits and where the audience comes from and the place you're giving the presentation.

Now, about the place, it should be well lit. If it's to dark then it signals that we should go to sleep.
The slides are easily to see, granted, but it's extremely difficult to see through closed eyelids.

% TODO: continue with cased and reasonaly populated.

\section{Telling Stories}

