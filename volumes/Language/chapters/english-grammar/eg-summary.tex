\chapter{English Grammar}

\section{B1-B2 Grammar}
\subsection{Capital Letters}
%https://learnenglish.britishcouncil.org/grammar/b1-b2-grammar/capital-letters-apostrophes
Days of the week, months, and holidays are capitalized, but seasons are not.
\begin{flushleft}
\setlength{\leftskip}{0.5cm} % Custom indent
\itshape
The days of the week are Monday, Tuesday, Wednesday, Thursday, Friday, Saturday, and Sunday.\\
My birthday is in September.\\
I celebrate Christmas in winter.\\
\end{flushleft}

Names of people and places are also capitalized.
\begin{flushleft}
\setlength{\leftskip}{0.5cm}
\itshape
The official residence of Prime Minister Keir Starmer is at 10 Downing Street.\\
The Earth rotates around the Sun.\\
United Kingdom is a country in Northwestern Europe.\\
\end{flushleft}

Words that come from names of places are capitalized.
\begin{flushleft}
\setlength{\leftskip}{0.5cm}
\itshape
Americans speak English.\\
Indian food is popular with Londoners.\\
The Duke of Westminster was born in London.
\end{flushleft}


\subsection{Apostrophes}
%https://learnenglish.britishcouncil.org/grammar/b1-b2-grammar/capital-letters-apostrophes
Contractions and possessions requires apostrophes.

For contractions the apostrophe is used to show where there are missing letters.
\begin{flushleft}
\setlength{\leftskip}{0.5cm}
\itshape
  It's cold, it won't get warm again until spring. (It's=It is; won't=will not)\\
  Don't cry, you'll be fine. (Don't=Do not; you'll=you will)\\
  She can't see; she's blind. (can't=can not; she's= she is)\\
\end{flushleft}
Note that its is a possession unlike it's which is a contraction of it is, or it has.

In possessions the apostrophe and the letter s comes after the noun to show that the noun owns something or someone.
\begin{flushleft}
\setlength{\leftskip}{0.5cm}
\itshape
My dog's favourite food is chicken.\\
George's parents live in Manchester.\\
Canadian's winters are cold.\\
The People's Republic of China.\\
\end{flushleft}

Plural nouns that end with an s, the apostrophe comes after the s.
\begin{flushleft}
\setlength{\leftskip}{0.5cm}
\itshape
My parents' house is large.\\
Our friend's cottage is cosy. She lives there by her self.\\
Our friends' cottage is cosy. They live there together.\\
\end{flushleft}


\subsection{Passives}
%https://learnenglish.britishcouncil.org/grammar/b1-b2-grammar/passives
The passive voice can be used to change the focus of the sentence.
\begin{flushleft}
\setlength{\leftskip}{0.5cm}
\itshape
  My scooter was stolen. (passive – focus on my scooter)\\
  Someone stole my scooter. (active – focus on someone)\\
\end{flushleft}


% Interested in the comma for relative clauses
%https://learnenglish.britishcouncil.org/grammar/b1-b2-grammar/relative-clauses-non-defining-relative-clauses





 

\section{C1 Grammar}
%https://learnenglish.britishcouncil.org/grammar/c1-grammar/advanced-passives-review

%https://learnenglish.britishcouncil.org/grammar/c1-grammar/avoiding-repetition-text

%https://learnenglish.britishcouncil.org/grammar/c1-grammar/contrasting-ideas

%https://learnenglish.britishcouncil.org/grammar/c1-grammar/possession-noun-modifiers


\subsection{Oxford Comma}


