\chapter{Fundamental Mathematics}

\section{Temenology}
\begin{itemize}
\item Axiom: ``TODO''~\cite{oed}.
\item Definition: ``TODO''~\cite{oed}.
\item Lemma: ``TODO''~\cite{oed}.
\item Theorem: ``TODO''~\cite{oed}.
\item Proposition: ``TODO''~\cite{oed}.
\item Corollary: ``TODO''~\cite{oed}.
\item Law: ``TODO''~\cite{oed}.
\end{itemize}


\section{Geometry}
\subsection{Volumes}
Volumes has a unite of cube, e.g., $m^3$ ``meter cube'', and a cube has a volume of $\text{lenght}\times\text{depth}\times\text{height} = \text{lenght}^3 = \text{depth}^3 = \text{height}^3 = \text{volume}$ since all sides are equal in a kube. For cuboid, however, the sides are different. A cube can express the voulume of three dimentional shapes.

\subsubsection{Pyramid}
Given a pyramid of height $h$, length $L$, and width $W$, the pyramid's voulume can be expressed in terms of cuboids of height $h/n$ where $n\to\infty$. The lenght of the cuboid layer is $m\times\frac{L}{n}$, where $m\in[1,\ldots,n]$. Likewise, the width is $m\times\frac{W}{n}$, which gives us the sum of all the layer cuboid making up the pyramid is equal to:
\begin{align*}
  &\sum_{m=1}^n\frac{h}{n}\times m\frac{L}{n}\times m\frac{W}{n} \\
  &=\frac{1}{n^3}hWL\sum_{m=1}^n m^2 \\
  &=\frac{1}{n^3}hWL\frac{n(n+1)(2n+1)}{6} \\
  &=\frac{1}{n^3}hWL\frac{2n^3+n^2+2n^2+n}{6} \\
  &=hWL\left(\frac{2n^3}{6n^3}+\frac{3n^2}{6n^3}+\frac{n}{6n^3}\right) \\
  &=hWL\left(\frac{1}{3}+\frac{1}{2n}+\frac{1}{6n^2}\right) \\
\end{align*}

Since $n\to\infty$:
\begin{equation*}
  \lim_{n\to\infty} hWL\left(\frac{1}{3}+\frac{1}{2n}+\frac{1}{6n^2}\right) = \frac{1}{3}hWL
\end{equation*}


\section{Irational numbers}
\subsection{Consant $e$}
\begin{equation*}
  e = \lim_{n\to\infty} \left( 1 + \frac{1}{n} \right)^n
\end{equation*}
\begin{equation*}
  e = 2 + \cfrac{1}{1+\cfrac{1}{2+\cfrac{1}{1+\cfrac{1}{1+\cfrac{1}{4+\cfrac{1}{1+\cfrac{1}{1+\cfrac{1}{6+\ldots}}}}}}}}
\end{equation*}



\subsection{Constant $\pi$}
