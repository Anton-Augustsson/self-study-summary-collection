
\section{Lecture 13: Equation of motion for simple harmonic oscillators}


As a quick refresher: we set $x = 0$ at the relaxed length of the spring, and extend it a distance $x$. The spring force is $-k x$, and the force we need to provide to overcome that is $+ k x$. The work we do in extending the spring is all stored as potential energy in the spring, so

\begin{equation}
U_{spring} = W = \int_0^x k x \mathop{dx} = \Big[\frac{1}{2} k x^2\Big]_0^x = \frac{1}{2} k x^2
\end{equation}

It follows, then, that $U_{spring} = 0$ at $x = 0$. As usual, we can define this how we want, but any other definition would only cause problems in most cases, and therefore be silly.

As is the case with gravity, as stated in the beginning of the lecture, the force is always in the direction \emph{opposite} that of increasing potential energy. For gravity, this turns out to be an always-attracting force. For springs, this turns out to be a restoring force: if you stretch the spring, the force is always such that it pulls to spring back together. If you instead compress the spring, the force reverses, and now tries to push it back to its original length. In both these cases, the force is in the opposite direction of increasing potential energy, since potential energy increases both when the spring is compressed and when it is extended.

Let's now look at the reverse situation. Can we go from knowing only the potential energy, to finding the spring's force? Yes, we can, and it's very easy: we take the derivative of the potential energy, with respect to $x$:

\begin{align}
U &= \frac{1}{2} k x^2\\
\frac{dU}{dx} &= + k x = - F_{sp}\\
\frac{dU}{dx} &= -F_x
\end{align}

Since the force is one-dimensional, we can write $F_x$ for the force. The minus sign is important, and as mentioned above: the force is in the direction \emph{opposite} the increase in potential energy. If $\frac{dU}{dx}$ is positive, you are moving in that direction, and you get a minus sign for the force -- it is opposite to our motion, since the motion is towards increasing potential energy.\\
If $\frac{dU}{dx}$ is negative, we are moving towards decreasing value of potential energy, and the force is positive (in the same direction as the $x$ motion).

In multiple dimensions, we can find a similar result. If we know the potential energy as a function of $x$, $y$ and $z$, we can find the force components along each axis by taking partial derivatives. So given $U(x, y, z)$, we can find the force components $F_x$, $F_y$ and $F_z$:

\begin{align}
\frac{\partial U}{\partial x} = - F_x\\
\frac{\partial U}{\partial y} = - F_y\\
\frac{\partial U}{\partial z} = - F_z
\end{align}

Two more realistic examples are covered in the lecture. First, for one-dimensional gravitational potential energy:

\begin{align}
U &= + m g y \text{ (with +y upwards)}\\
\frac{dU}{dy} &= m g\\
F_y = -\frac{dU}{dy} &= - mg
\end{align}

And indeed, the gravitational force is $- m g$, assuming increasing $y$ is upwards.

Next, another one-dimensional problem of gravitational potential energy, this time in general, rather than very close to Earth's surface:

\begin{align}
U &= -\frac{M m G}{r} \text{ (where $M$ is the mass of the Earth)}\\
\frac{dU}{dr} &= +\frac{M m G}{r^2}\\
F_r = -\frac{dU}{dr} &= -\frac{M m G}{r^2}
\end{align}

Again, we find a familiar result.

\subsection{Stable and unstable equilibrium}

Next up, let's look at equilibrium. Say we have a surface, that may look like this:

\begin{figure}[H]
  \centering
\begin{tikzpicture}[scale=1]

  \draw[thick, smooth] 
  (0,1) 
  to[out=-60, in=180] (0.5,0.5)
  to[out=0, in=180] (1.5,2)
  to[out=0, in=180] (2.5,1)
  to[out=0, in=180] (4,3)
  to[out=0, in=180] (6,0)
  to[out=0, in=180] (7,1)
  to[out=0, in=180] (8,0.5)
  to[out=0, in=180] (8.5,0.5);

  \draw[dotted] (0,0) -- (8,0) node[right] {$y$} node[at start, left] {$u=0$};

  \filldraw[black] (1.5,2) circle (2pt);
  \filldraw[black] (2.5,1) circle (2pt);
  \filldraw[black] (4,3) circle (2pt);
  \filldraw[black] (6,0) circle (2pt);
  \filldraw[black] (7,1) circle (2pt);

  \draw[thick, orange, -{Stealth[scale=1.5]}] (1.5,2) -- (1.5,0.5) node[right=2mm] {$mg$};
  \draw[thick, orange, -{Stealth[scale=1.5]}] (1.5,2) -- (1.5,3.5) node[right=2mm] {$N$};

  \draw[-{Stealth[scale=1.5]}] (-1,2) -- (-1,3) node[right=1mm] {$y$};
  \draw[-{Stealth[scale=1.5]}] (-1,2) -- (0,2) node[above=1mm] {$x$};

\end{tikzpicture}
\end{figure}



The points along the curve have a gravitational potential energy which is $U = m g y$, since we defined $U = 0$ at $y = 0$ (the $= 0$ part is just outside the screen grab above). Since the plot is of $y = f(x)$, for some function $f$, we can also write that $U = m g f(x)$.

There are then points along this curve where $\displaystyle \frac{dU}{dx} = 0$. Those points occur where the curve's slope (or derivative) are zero, by definition, which is at the top of each peak, and at the bottom of each valley, as signified by a dot in the above figure.

From the definition we found before, that then means that $-F_x = 0$, so the net force in the $x$ direction is zero.\\
At such a point, there is a force $-m g$ in the $y$ direction, and a normal force $N = + m g$, so that there is no net force there, either.

Since there is no net force on the object at one of these points, and we can put it at rest, it will stay exactly where it is.\\
However, there is an important difference between these two types of points  (peaks and valleys). If we try to balance a marble at the top of such a peak, just about any tiny vibration, small amount of wind etc. will get it moving. Being at the top of a large downwards slope, in either direction, it will then clearly begin to accelerate downwards -- and again, the force is in the direction of decreasing potential energy (which of course is the same thing as being in the \emph{opposite} direction as \emph{increasing} potential energy).

However, if we put a marble in one of the valleys, what happens? If there is a small force, causing a motion in any direction, it will be forced back into the valley. The force is yet again in the direction opposing the increase in potential energy, and potential energy increases both to the left \emph{and} to the right! Therefore, the force is such that the marble is returned to the middle of the valley again, to the point of lowest potential energy.

The difference between these two zero points are that the peaks provide \emph{unstable equilibrium}, while the valleys provide \emph{stable equilibrium}. If there is a disturbance in the first case, it goes out of control. In the second case, in the valleys, any small disturbance is automatically countered, and the object goes back to where it was, at the bottom.

We can find out which of these two cases a point is, mathematically, by looking at the second derivative. If the second derivative of potential energy with respect to $x$ is positive, it's a point of stable equilibrium. If it is negative, it's instead a point of unstable equilibrium.

\subsection{Another look at a spring oscillator}

Let's have another look at the oscillation of a mass on a spring, this time from an energy perspective. We know that $\displaystyle U = \frac{1}{2} k x^2$, so a plot of $U(x)$ would be a parabola:

\begin{figure}[H]
  \centering
\begin{tikzpicture}[scale=1]

  \draw[thick] (0,-1) -- (0,4) node[left] {$u$};
  \draw[thick] (-3,0) -- (3,0) node[right] {$x$};

  \fill (-1.5,0) circle (0.5mm) node[below] {$-x_{max}$};
  \fill (0.75,0) circle (0.5mm) node[below] {$x$};
  \fill (1.5,0) circle (0.5mm) node[below] {$x_{max}$};

  \draw[dotted] (1.5,0) -- (1.5,3);
  \draw[dotted] (0.75,0) -- (0.75,1);

  \draw[black, line width = 0.50mm] plot[smooth,domain=-2:2] (\x, {(\x)^2}) node[right] {$\frac{1}{2}kx^2$};

  \draw[orange, -{Stealth[scale=0.8]}] (0.75,-0.35) -- (0.2,-0.35) node[midway, below] {$v$};
  \draw[orange, -{Stealth[scale=0.8]}] (0.75,-0.35) -- (1.2,-0.35);

\end{tikzpicture}
\caption{}
\end{figure}


Say we have a mass attached to a spring, as usual, and we extend it to $x_{max}$, and let it go with zero speed.

We know that it will oscillate between $+x_{max}$ and $-x_{max}$, but we can now gain a second insight into this oscillation (albeit one mentioned earlier). Say we release the mass at an extension $x_{max}$ beyond the spring's natural length. That means the potential energy in the spring at that time is

\begin{equation}
U_{initial} = \frac{1}{2} k x_{max}^2
\end{equation}


Since we know that the force will be in the direction opposing the increase in potential energy, the mass will be pulled inwards, towards $x = 0$. Once it crosses the zero point, the force switches directions, since the current \emph{velocity} vector is towards increasing potential energy (the spring is being compressed to be shorter than its natural length). That means the force (and thus the acceleration) instantly flips over, and the mass starts slowing down. The new force is once again in the direction opposing the increase in potential energy, which is again towards $x = 0$, which is now towards the right in the figure.

Because spring forces are conservative (for ideal springs), we can use conservation of energy to write an equation for this system. The \emph{total} energy in the system must equal the spring's stored potential energy at $t = 0$, plus the mass's kinetic energy at $t = 0$. The latter is zero, since we release it at rest (zero speed), so $E_{total} = U_{initial}$. That energy must be held constant -- conservation of energy. Therefore, the sum of the mass's kinetic energy $\displaystyle \frac{1}{2} m v^2$ and the spring's stored energy $\displaystyle \frac{1}{2} k x^2$ must always equal that initial energy. We can set up an equation for this:

\begin{equation}
\frac{1}{2} m v^2 + \frac{1}{2} k x^2 = \frac{1}{2} k x_{max}^2
\end{equation}

This equation must \emph{always} hold for this system, unless there are other forces, such as friction, which we ignore for now.\\
Because $v = \dot{x}$, we can rewrite this equation, by making that substitution, and getting rid of all the one-halves, and dividing by $m$:

\begin{align}
\frac{1}{2} m \dot{x}^2 + \frac{1}{2} k x^2 = \frac{1}{2} k x_{max}^2\\
\dot{x}^2 + \frac{k}{m} x^2 - \frac{k}{m} x_{max}^2 = 0
\end{align}

We can then take the time derivative of this. Keep in mind that since the equation is in terms of $x$, we need to use the chain rule for most terms.

\begin{align}
\frac{d}{dt} \left(\dot{x}^2 + \frac{k}{m} x^2 - \frac{k}{m} x_{max}^2\right) = \frac{d}{dt} (0)\\
2 \dot{x} \ddot{x} + 2 \frac{k}{m} x \dot{x} - 0 = 0
\end{align}

We can simplify this equation by dividing by $2 \dot{x}$:

\begin{equation}
\ddot{x} + \frac{k}{m} x = 0
\end{equation}

Isn't it remarkable? We get the equation for simple harmonic motion, and so we find the same old solutions:

\begin{align}
x        &= x_{max} \cos(\omega t + \varphi)\\
\dot{x}  &= - \omega x_{max} \sin(\omega t + \varphi)\\
\ddot{x} &= - \omega^2 x_{max} \cos(\omega t + \varphi) = -\omega^2 x\\
\omega  )    &= \sqrt{\frac{k}{m}}\\
T           &= \frac{2 \pi}{\omega} = 2 \pi \sqrt{\frac{m}{k}}
\end{align}

\subsection{Motion of a ball along a circular track}

Say we have a circular (or at least semicircular) track of radius $R$. We define $y = 0$ and $U = 0$ to be at the bottom of the track.

\begin{figure}[H]
  \centering
\begin{tikzpicture}[scale=1]

  \draw (0,0) arc (180:360:30mm);
  \draw (3,-0.62) arc (270:340:5mm) node[below=2mm] {$\theta$};
  \draw (3,0) -- (5.5,-1.65) node[midway, above] {$R$};
  \draw[dashed] (3,-1.65) -- (5.5,-1.65);
  \draw[dotted] (3,0) -- (3,-3);

  \fill (3,-3) circle (0.5mm);
  \fill (5.5,-1.65) circle (0.5mm);

  \draw[dotted] (0,-3) -- (6,-3) node[right] {$y=0$} node[at start, left] {u=0};
  \draw[-{Stealth}] (5.5,-3) -- (5.5,-1.65) node[right] {$y$};

  \draw[decorate, decoration = {brace, amplitude=1.5mm}] (3,-1.65) -- (3,0) node[midway, left=2mm] {$R\cos{\theta}$};
  \draw[decorate, decoration = {brace, amplitude=1.5mm}] (3,-3) -- (3,-1.65) node[midway, left=2mm] {};

\end{tikzpicture}
\caption{}
\end{figure}


When the ball is at some random location $y$, we can find the angle made with the vertical, $\theta$, via trigonometry.\\
First, we find that the radius $R$ acts as the hypotenuse of a right triangle, where the $x$ component $R \sin \theta$ is at the bottom, and the left side has height $R \cos \theta$. Note that the $y$ coordinate fits $y = R - R\cos \theta$, so that $y = R (1 - \cos \theta)$.

With that in mind, we can write $U$ as a function of the angle $\theta$ now:

\begin{equation}
U = m g y = m g R (1 - \cos \theta)
\end{equation}

Notice that at $\theta = 0$, $U = 0$, as we defined.\\
At $\displaystyle \theta = \frac{\pi}{2}$, $U = m g R$, since it is a height $R$ above $y = 0$.

Using the definition of a radian as the arc length subtended by an angle, where $dS$ is the arc length and $d\theta$ the angle, we find

\begin{align}
\frac{dS}{R} &= d\theta\\
dS &= R d\theta
\end{align}

Taking the time derivative of both sides, we find

\begin{equation}
\frac{dS}{dt} = R \frac{d\theta}{dt} = R \dot{\theta}
\end{equation}

The left-hand term is just the distance moved per unit time, so $\displaystyle \frac{dS}{dt} = v = R \dot{\theta}$.

In most cases, we use $\displaystyle \omega = \dot{\theta} = \frac{d\theta}{dt}$, but in most cases, $\omega$ is also a constant. In this case, it is a function of the angle; the angle will change the fastest near $\theta = 0$ (at the bottom), where the speed is at a maximum, while it will change slower as the ball climbs up the ``edges'' of the circle (I think of it as a ``two-dimensional bowl''), as it is about to come to a halt, and change direction.

As a short aside, we can, as a small-angle approximation, use

\begin{equation}
\cos \theta \approx 1 - \frac{\theta^2}{2}
\end{equation}

This approximation uses the first two terms of a Taylor expansion for $\cos \theta$. 

Last time we used such an approximation, we used only the first term, $\cos \theta \approx 1$. That's too inexact for this case, though -- we would end up with $U = m g R (1 - 1) = 0$ for all $\theta$!

Even for angles of about 11.5 degrees, the error caused by this approximation is way, way less than 1\% (less than 0.01\%, actually). In fact, for as much as 30 degrees, we have $\cos(\ang{30}) \approx 0.8660254$, while the approximation gives $0.862922$. It's off by about 0.3\% -- still not a lot, all things considered.

Let's return to the problem at hand. Using this approximation, we apply the conservation of mechanical energy to this system. The total mechanical energy must be a constant. If we release the object as zero speed, and thus zero kinetic energy, the total energy (kinetic + potential) must always equal:

\begin{equation}
M_E = \frac{1}{2} m v^2 + m g R(1 - \cos \theta)
\end{equation}

Since $v = R \dot{\theta}$, $v^2 = R^2 \dot{\theta}^2$. Let's also apply our approximation for the cosine. What we end up with is

\begin{align}
M_E &= \frac{1}{2} m R^2 \dot{\theta}^2 + m g R(1 - (1 - \frac{\theta^2}{2}))\\
M_E &= \frac{1}{2} m R^2 \dot{\theta}^2 + m g R \frac{\theta^2}{2}
\end{align}

We can now take the time derivative of this. $M_E$ is a constant, so that becomes zero. As far the rest, we use the chain rule again:

\begin{align}
0 &= \frac{1}{2} m R^2 (2 \dot{\theta} \ddot{\theta}) + \frac{m g R}{2} 2 \theta \dot{\theta}\\
0 &= m R^2 \dot{\theta} \ddot{\theta} + m g R \theta \dot{\theta}\\
0 &= R^2 \ddot{\theta} + g R \theta
\end{align}

We can rearrange that as

\begin{equation}
\ddot{\theta} + \frac{g}{R} \theta = 0
\end{equation}

... and it is then again obvious that we have as simple harmonic oscillator! We know the solution to this differential equation, so we can write down

\begin{align}
\theta &= \theta_{max} \cos (\omega t + \varphi)\\
\omega &= \sqrt{\frac{g}{R}}\\
T      &= 2 \pi \sqrt{\frac{R}{g}}
\end{align}

Note that this $\omega$ is completely unrelated to the $\displaystyle \frac{d\theta}{dt}$ we had earlier in the derivation -- it's a good thing we didn't call that $\omega$! This one is a constant, while the other one changed with time.

Note how these equations are identical to the ones for a pendulum, that we derived earlier, also using a small-angle approximation. This time, however, it is our approximation which caused the similarity -- we made the equation quadratic in $\theta$ by doing that. The spring oscillation was quadratic in $x$ from the beginning.

Finally, on to an important detail. Nowhere in this derivation did we consider the normal force from the track on the ball. Is it really safe to ignore it? Why?

It turns out that yes, we can ignore it, because in the case of this circular track, it is always perpendicular to the direction of motion. A force perpendicular to a motion \emph{cannot} do work, because of the definition of the dot product: an angle of $\ang{90}$ between force and displacement always means zero work.


\section{Lecture 14: Orbits and escape velocity}

We can find the \textit{escape velocity} for a given body (such as the Earth) quite easily, using conservation of energy. The kinetic energy at launch must be $\displaystyle \frac{1}{2} m v_{esc}^2$, and since the problem definition is that in never adds to that kinetic energy (no engines). That must therefore be the total energy of the object, at all times. The total energy at any given time is the sum of the kinetic energy at some point $r$, which we call $\displaystyle \frac{1}{2} m v_r^2$, and the potential energy at that point, $\displaystyle - \frac{G M m}{r}$, with $M$ being the mass of the Earth (or the body), and $m$ the mass of the object trying to escape.

\begin{equation}
\frac{1}{2} m v_{esc}^2 + \left(-\frac{G M m}{R_{Earth}}\right) = \frac{1}{2} m v_r^2 + \left(-\frac{G M m}{r}\right)
\end{equation}

On the left side, we have the total energy as we start out our journey, and on the right, the total energy some distance $r$ away.

However, since the goal is for the energy to be enough to escape to an infinite distance, the kinetic energy ``at'' infinity (let's just say extremely, extremely far away, since being ``at'' infinity is meaningless), the potential energy is zero, by definition. The kinetic energy is also zero, \emph{if} we gave it \emph{just} enough energy, and not any more than required (we know that the Earth's gravity will reduce the speed, and thus the kinetic energy, as time goes on).

Because of this, we can set the entire right side of the equation equal to zero, which is valid ``at'' infinity (or just so far away that the gravitational pull of the Earth is now completely negligible), and solve for the escape velocity:

\begin{align}
\frac{1}{2} m v_{esc}^2 -\frac{G M m}{R_{Earth}} = 0\\
v_{esc}^2 -\frac{2 G M}{R_{Earth}} = 0\\
v_{esc} = \sqrt{\frac{2 G M}{R_{Earth}}}
\end{align}

where, again, $M$ is the mass of the Earth. For Earth, this value is then about 11.2 km/s. So if we neglect air resistance, which will surely make these results valid if we are at the Earth's surface, if we could fire a cannon ball at more than 11.2 km/s, it would never fall back to Earth.

If the initial velocity is greater, then you will still have kinetic energy (and thus speed) left when you've escaped. In the case you do ``escape'', with the condition $E_{init} \ge 0$, you are in an \emph{unbound orbit}. In the case that $E_{init} < 0$, you enter a \emph{bound orbit}, and will never escape the gravitational pull of the Earth (or the body in question).

\subsection{Circular orbits}

Say we have a mass $m$ orbiting the Earth, with Earth's mass being $M$, and say that $m \ll M$.\\
It moves in a circle around the Earth at constant (tangential) speed, but not constant velocity -- there is a constant centripetal acceleration, or it wouldn't be moving in a circle. Centripetal acceleration is provided by centripetal force, which in this case is the attractive force of gravity of the Earth on the mass.

We know how to find the gravitational force using the Newton's law of universal gravitation, and we can set that equal to the centripetal force $\frac{m v^2}{r}$ (which is just $a_c m$, via $F = m a$):

\begin{align}
\frac{G M m}{r^2} = \frac{m v_{orbit}^2}{r}\\
\frac{G M}{r} = v_{orbit}^2\\
\sqrt{\frac{G M}{r}} = v_{orbit}
\end{align}

where $r$ is the radius of the orbit, which has nothing to do with the radius of the Earth itself. $v_{orbit}$ is then the tangential speed of the object that is in orbit. Knowing these facts, we can now find the period of the orbit:

\begin{equation}
T = \frac{2 \pi r}{v_{orbit}} = 2 \pi \frac{r^{3/2}}{\sqrt{G M}}
\end{equation}

If we plug in the Sun's mass, and $r = \SI{149.6e9}{m}$, the approximate average distance to the sun, we find Earth's orbital period $T \approx 365.33$ days. Not bad at all, since this is only an approximation (it ignores the several things that matter, including the Earth's elliptical orbit).

As a different example, we can take the space shuttle, or the space station, which orbit at 250-400 km above the Earth's surface. If we make the calculation for 400 km, so that $r = R_{Earth} + 400$ km, we find $v_{orbit} \approx \SI{8}{km/s}$ and $T \approx 90$ minutes(!).

Note that the orbital parameters are independent on the mass of the orbiting object. It only depends on the mass of the object you orbit, and the distance from it (i.e. the radius of the orbit), times some constants.

Also note that $v_{esc} = \sqrt{2} \times v_{orbit}$, for a given point. (In $v_{esc}$, we used the radius of the Earth, because we wanted to calculate the escape velocity from the surface.)

The total mechanical energy at some radius $r$, at orbital velocity $v_{orbit}$, is

\begin{equation}
E = \frac{1}{2} m v_{orbit}^2 - \frac{G M m}{r}
\end{equation}

We can substitute the value for $v_{orbit}^2$ in there, though:

\begin{align}
E = \frac{1}{2} m \frac{G M}{r} - \frac{G M m}{r}\\
E = -\frac{1}{2} \frac{G M m}{r} = \frac{1}{2} U = - K_E
\end{align}

Quite an interesting result. In words, then, the total energy of an orbiting object is always half its gravitational potential energy, and also the negative of its kinetic energy.

Now, for something completely different (more on orbits in a few weeks).

\subsection{Power}

\emph{Power} is energy per unit time -- or work per unit time, since energy and work are closely related, and share the same dimension. The SI unit for power is joules per second, or watts, W; not to be confused with W for the quantity of work! If we have $W = $ something then it's work; if we have $P = 10$ W, then it's watts.

Stated differently, it is then just the derivative of work -- that is, $P = \displaystyle \frac{dW}{dt}$.\\
Since $dW = \vec{F} \cdot \vec{dr}$, we can take the time derivative of both sides:

\begin{equation}
\frac{dW}{dt} = \vec{F} \cdot \frac{\vec{dr}}{dt}
\end{equation}

... and since the rate of change versus time of $\vec{dr}$ is simply the velocity:

\begin{equation}
P = \vec{F} \cdot \vec{v}
\end{equation}

\subsubsection{Power in riding a bicycle}

Let's look at an example: riding a bicycle. We try to keep a constant velocity, which means there should be no net force on the bike. However, there \emph{is} air drag, and the force opposing your motion, $F_{res} \propto k v^2$. In order for there to be no \emph{net} force, your pedaling must then provide an equally great force in the forwards direction, in order for you to keep a constant speed.

As an aside, how does pedaling provide this force? First, you push down on the pedals, and the pedals push back on you with equal force via Newton's third law. This causes no net force on the bike, and we call these forces \emph{internal forces}.

The pedals push on the chain, and the chain pushes on the wheel, all of which cancels, but finally, the wheel now wants to rotate, because of the force exerted by the chain.

The wheel pushes backwards on the ground, which leads to a reaction force such that the ground pushes the wheel forward. Finally something useful! This only works because of friction, of course -- without friction, it would simply start spinning, and there would be no forward force on the bicycle.

Now, let's look at the amount of power you must provide to overcome air resistance. We can model this as a regime II problem, so the drag force is proportional to $k_2 v^2$. Say that the power you must provide at 10 miles/hour is 15 watts -- which is a given, and not something we actually show.

Now, the power we must provide is $P = \vec{F} \cdot \vec{v}$, as we showed earlier. Since the force and the velocity are in the same direction, $P = F v$. Since $F = k_2 v^2$, we find $P = k_2 v^3$! It is proportional not to $v^2$, but $v^3$.

If you then want to speed up to 25 mph, 2.5 times the original speed, you need to provide $2.5^{3} \approx 15.6$ times the power, about 230 watts! For 30 mph, 3 times the original speed, you need $3^3 = 27$ times the power (over 400 watts)! Needless to say, we reach the limits of human physiology rather quickly if we keep going like this. At 50 mph, it would take over 240 times the power (over 1800 watts -- far above what any human could do, except for elite athletes for a period of seconds or less)!

\subsection{Heat energy}

First, a few definitions. We use the symbol $Q$ for heat energy, often in the unit of calories. A calorie is the energy required to raise the temperature of 1 gram of water by 1 degree Centigrade (or 1 Kelvin, which is the same thing). There are, unfortunately, a ton of different definitions for a calorie, but all are close to 4.2 joules. (Some are defined as the energy required to heat 1 g of water from $3.5 {}^\circ$C to $4.5 {}^\circ$C, others from 14.5 to 15.5, 19.5 to 20.5, etc.)

Next, there is the \emph{specific heat} $C$, which is a constant (for a given material) that specifies the amount of energy required to raise the temperature of that material by 1 degree centigrade, per unit mass. That is, it's in $\text{cal}/(g {}^\circ\text{C})$.\\
If we want the unit to use kilograms instead of grams, which is always a nice thing when using the MKS (meter-kilogram-second) system, we simply multiply the constant by 1000.

The amount of heat energy Q, is then

\begin{equation}
Q = m C \Delta T
\end{equation}

in calories, if $m$ is in grams, $C$ in the units stated above (per gram, not per kilogram), and $\Delta T$ in either Kelvin or degrees centigrade (they are equivalent; the zero point is the only difference).\\
As a reference, ice has a specific heat of about 0.5, compared to liquid water's 1. Aluminium has a specific heat of about 0.2, and lead a very low 0.03.

James Joule first found that mechanical work and heat energy are equivalent in 1845, though he was not the first to begin research on the topic. This research, among other things, led to the naming of the joule in his honor.

