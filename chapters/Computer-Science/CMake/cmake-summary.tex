\chapter{CMake}

\section{Introduction}
CMake is a cross-platform, open-source build system generator. It generates native build configurations (e.g., \texttt{Makefiles}, \texttt{ninja}, Visual Studio projects) to build, test, and package software. It supports both in-source and out-of-source builds and is highly configurable.

\section{Core Concepts}
\begin{itemize}
    \item \textbf{CMakeLists.txt:} The main configuration file used to define the build process.
    \item \textbf{Targets:} Executables or libraries generated by the build process.
    \item \textbf{Generator:} A tool that CMake uses to produce native build scripts (e.g., \texttt{Unix Makefiles}, \texttt{Ninja}).
    \item \textbf{Out-of-source Build:} Keeping build artifacts separate from source code.
\end{itemize}

\section{Basic Commands in \texttt{CMakeLists.txt}}
\begin{description}
    \item[\texttt{cmake\_minimum\_required(VERSION X.Y)}] Specifies the minimum version of CMake required.
    \item[\texttt{project(NAME LANGUAGES ...)}] Defines the project name and supported languages (e.g., C, C++).
    \item[\texttt{add\_executable(target source1 ...)}] Defines an executable target.
    \item[\texttt{add\_library(target source1 ...)}] Defines a library target.
    \item[\texttt{target\_include\_directories(target ...)}] Adds include directories to a target.
    \item[\texttt{target\_link\_libraries(target ...)}] Links libraries to a target.
    \item[\texttt{find\_package(NAME ...)}] Locates external dependencies (e.g., Boost, OpenCV).
    \item[\texttt{install(...)}] Specifies installation rules for targets or files.
    \item[\texttt{option(NAME "Description" DEFAULT)}] Defines a build-time configuration option.
    \item[\texttt{if(condition)}] Adds conditional logic.
    \item[\texttt{set(NAME VALUE)}] Sets variables for configuration or build logic.
\end{description}

\section{Example CMake Project}
\subsection{Directory Structure}
%\begin{verbatim}
%project/
%├── CMakeLists.txt
%├── src/
%│   ├── main.cpp
%├── include/
%│   ├── my_header.h
%\end{verbatim}

\subsection{CMakeLists.txt}
\begin{verbatim}
cmake_minimum_required(VERSION 3.15)
project(MyProject LANGUAGES CXX)

# Set C++ standard
set(CMAKE_CXX_STANDARD 17)

# Add include directories
include_directories(include)

# Add executable target
add_executable(my_app src/main.cpp)
\end{verbatim}

\section{Build Steps}
\begin{enumerate}
    \item Create a build directory:
    \begin{verbatim}
    mkdir build && cd build
    \end{verbatim}
    \item Run CMake to configure the project:
    \begin{verbatim}
    cmake ..
    \end{verbatim}
    \item Build the project:
    \begin{verbatim}
    cmake --build .
    \end{verbatim}
\end{enumerate}

\section{Advanced Features}
\subsection{Variables and Options}
\begin{itemize}
    \item Use \texttt{set()} to define variables:
    \begin{verbatim}
    set(VARIABLE_NAME value)
    \end{verbatim}
    \item Define options with \texttt{option()}:
    \begin{verbatim}
    option(BUILD_TESTS "Build the test suite" ON)
    \end{verbatim}
\end{itemize}

\subsection*{Dependency Management}
\begin{itemize}
    \item Use \texttt{find\_package()} to locate external dependencies:
    \begin{verbatim}
    find_package(Boost REQUIRED COMPONENTS system filesystem)
    target_link_libraries(my_app Boost::system Boost::filesystem)
    \end{verbatim}
    \item Use \texttt{FetchContent} to manage dependencies:
    \begin{verbatim}
    include(FetchContent)
    FetchContent_Declare(
        googletest
        URL https://github.com/google/googletest/archive/release-1.10.0.zip
    )
    FetchContent_MakeAvailable(googletest)
    \end{verbatim}
\end{itemize}

\subsection*{Cross-Compilation}
CMake supports cross-compilation by specifying a toolchain file with \texttt{-DCMAKE\_TOOLCHAIN\_FILE}:
\begin{verbatim}
cmake -DCMAKE_TOOLCHAIN_FILE=toolchain.cmake ..
\end{verbatim}

\section*{Generators}
CMake provides support for different build tools via generators:
\begin{itemize}
    \item \texttt{Unix Makefiles}: Generates \texttt{Makefiles} for \texttt{make}.
    \item \texttt{Ninja}: Generates build files for \texttt{ninja}.
    \item \texttt{Visual Studio}: Generates Visual Studio project files.
\end{itemize}
Choose a generator with the \texttt{-G} flag:
\begin{verbatim}
cmake -G "Ninja" ..
\end{verbatim}

\section*{Best Practices}
\begin{itemize}
    \item Use \textbf{out-of-source builds} to keep build artifacts separate from source code.
    \item Modularize your build system with subdirectories and subprojects.
    \item Always specify a \texttt{CMakeLists.txt} in each subdirectory.
    \item Use \texttt{target\_include\_directories()} instead of global include directories.
\end{itemize}

\section*{References}
\begin{itemize}
    \item \href{https://cmake.org/documentation/}{CMake Documentation}
    \item \href{https://cliutils.gitlab.io/modern-cmake/}{Modern CMake Guide}
\end{itemize}